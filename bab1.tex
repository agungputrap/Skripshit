
%-----------------------------------------------------------------------------%
\chapter{\babSatu}
%-----------------------------------------------------------------------------%
Bab ini menjelaskan latar belakang, permasalahan, tujuan dan ruang lingkup 
penelitian, serta sistematika penulisan tugas akhir penelitian 

%-----------------------------------------------------------------------------%
\section{Latar Belakang}
%-----------------------------------------------------------------------------%

\hspace{0.5 cm}Perjalanan kehidupan manusia dipenuhi dengan berbagai kejadian. Manusia memerlukan usaha untuk dapat memenuhi kebutuhan dan menjaga eksistensi dari ancaman, baik itu dari dalam dan dari luar. Salah satu ancaman tersebut adalah penyakit. Perkembangan ilmu dan daya pikir manusia hanya bisa menyembuhkan tetapi tidak dapat menghapus penyakit tersebut. Muncul jenis penyakit baru yang juga memerlukan pengetahuan baru dalam mencegah penyebaran penyakit tersebut. Berbagai pengobatan dan kepecayaan dalam menyembuhkan penyakit bermunculan. Negara - negara Asia timur muncul dengan pengobatan herbal dan teknik akupuntur. Arkeolog menemukan bukti bahwa pemanfaatan tanaman obat \ herbal telah ada sejak 60.000 tahun silam (masa \textit{Paleolithic}). Tulisan berumur 5.000 tahun tentang daftar tanaman obat juga ditemukan. Tulisan ini merupakan daftar herbal yang dibuat oleh penduduk Sumeria. Tidak hanya Sumeria, dalam sejarah Mesir, Yunani maupun Cina terdaftar sebagai negara yang mengembangkan pengobatan herbal. Cina tidak hanya mengembangkan herbal, mereka juga mengembangkan teknik akupuntur dalam menyembuhkan penyakit.

Seiring dengan berkembang ilmu pengetahuan dan teknologi manusia khususnya \textit{proteomics} dan \textit{genomics}, manusia dapat mempelajari penyakit pada tingkat molekul. Hal ini memberikan pengetahuan dan kepastian dalam pengembangan obat secara modern. \textit{Drug discovery} merupakan runtutan skenario panjang dalam menemukan calon obat yang berpotensi. Sebelum masuk dalam \textit{drug discovery}, dilakukan \textit{pre-drug discovery}. Dalam tahap ini peneliti mempelajari :
\begin{itemize}
	\item Karaktersitik penyakit tersebut
	\item Bagaimana penyakit tersebut mempengaruhi gen penderita
	\item Bagimana gen yang terkena tersebut memproduksi protein
	\item Bagaimana protein tersebut berinteraksi dengan lingkungan sekitar
	\item Bagaimana protein tersebut mempengaruhi lapisan dimana protein tersebut berperan 
	\item Bagaimana penyakit tersebut mempengaruhi \textit{host}(tubuh yang terkena penyakit)
\end{itemize}
 \hspace{0.5 cm}Dengan berbekal pengetahuan dari proses \textit{pre-drug discovery} tersebut, peneliti perlu memastikan bagian yang berupa molekul tunggal (gen/protein) yang berpengaruh dalam penyakit tersebut. Setelah hal tersebut, perlu dilakukan validasi dengan cara menyuntikan "target" tersebut ke dalam sel hidup atau \textit{speciment} percobaan. Hal ini untuk memastikan bahwa molekul target yang diambil adalah valid. Setelah molekul target valid, peneliti perlu mencari molekul yang dapat membalikan kerja molekul target. Diperlukan waktu yang cukup lama untuk mengidentifikasi molekul pembalik tersebut dan memastikan bahwa molekul tersebut dapat digunakan sebagai obat. Berdasarkan survey, waktu yang diperlukan dalam suatu \textit{drug discovery} sekitar 10-15 tahun. Selain itu biaya yang dikeluarkan juga sangat besar (800 juta - 1 milyar dollar) dan kemungkinan gagal dalam menemukan obat baru yang berpotensi pun tetap ada.         

Waktu yang dibutuhkan, biaya yang dikeluarkan, serta kemungkinan gagal dalam menemukan obat baru yang berpotensi menjadi perhatian utama dalam \textit{drug discovery} mula - mula. Komputer turut mengambil andil dalam \textit{drug discovery} modern. Pemanfaatan komputer tersebut menggantikan proses uji coba pencarian molekul yang dapat membalikan efek dari molekul target yang sebelumnya dilakukan dengan eksperimen terhadap makhluk hidup digantikan dengan simulasi. \textit{Virtual screening}, yaitu mencari molekul yang dapat mengikat (\textit{binding}) dengan molekul target dari \textit{database} molekul yang tersimpan. Untuk memastikan molekul terikat tersebut, \textit{virtual screening} memanfaatkan perhitungan fungsi/\textit{scoring} untuk menentukan mana molekul yang "\textit{best-fit}" dengan molekul target. Pada umumnya \textit{virtual screening} tersebut berjalan secara otomatis dengan menggunakan program komputer. Dengan cara tersebut, para peneliti dapat memangkas waktu dan biaya yang dibutuhkan dalam \textit{drug discovery}. Beberapa aplikasi \textit{virtual screening} yang dapat digunakan antara lain : Auto Dock, DOCK, Gold, V Life MDS, Flex X.

\textit{Virtual screening} yang mencakup evaluasi seluruh koleksi \textit{database} molekul terhadap molekul target membutuhkan tenaga komputasi yang tidak sedikit. Tidak hanya itu, kapasitas memori penyimpanan untuk koleksi data molekul maupun hasil \textit{virtual screening} yang cukup besar, sumber daya listrik yang digunakan untuk menjaga komputer tetap bekerja juga diperlukan. Kemampuan komputasi pada komputer terletak pada \textit{processor}. Komputer akan mengenali \textit{virtual screening} yang dijalankan sebagai kumpulan instruksi yang harus dikerjakan oleh \textit{processor}. Semakin cepat \textit{processor}, semakin banyak instruksi yang dapat dikerjakan setiap detiknya. \textit{Virtual screening} dapat dilakukan pada sebuah komputer yang memiliki koleksi data molekul yang besar, namun masih terdapat kendala, seberapa lama waktu yang dibutuhkan untuk memproses semua koleksi data molekul tersebut. Untuk memenuhi kebutuhan tenaga komputasi tersebut, dibutuhkan infrastruktur komputer yang dapat saling bekerja sama untuk melakukan suatu komputasi. Infrastruktur ini dikenal sebagai \textit{supercomputer}. \textit{Supercomputer} terdiri dari beberapa \textit{processor} yang memiliki kemampuan komputasi yang tinggi. Komputer tersebut dapat tersebar diberbagai tempat dan dihubungkan dengan jaringan atau kumpulan komputer yang diletakan saling berdekatan pada suatu tempat. Proses kerja \textit{supercomputer} adalah \textit{centralization} dimana setiap \textit{processor} mengerjakan \textit{task} yang sama dan hasilnya akan kembali diolah oleh suatu \textit{processor} untuk menghasilkan output akhir. Konsumsi energi yang dibutuhkan oleh \textit{supercomputer} tergolong sangat besar dan sebagian besar hasil dari eksekusi \textit{task} adalah panas. Dibutuhkan biaya tambahan untuk pemeliharaan \textit{supercomputer} agar dapat terus bekerja pada suhu yang terjaga.

Besarnya biaya yang diperlukan oleh \textit{supercomputer} merupakan kendala utama. \textit{Grid computing} memberikan alternatif lain untuk permasalahan tersebut. Teknologi ini memanfaatkan sekumpulan komputer biasa secara fisik yang saling terhubung melalui suatu jaringan sebagai suatu kesatuan komputer (\textit{super virtual computer}). Tidak seperti \textit{supecomputer}, setiap komputer bekerja sama untuk menyelesaikan sebuah \textit{task}, yang kemudian \textit{task} tersebut akan hilang jika selesai dan tergantikan dengan \textit{task} yang baru. Masing masing komputer penyusun tersebut akan bekerja secara parallel. Permasalahan kembali muncul ketika pengadaan komputer secara fisik yang banyak dan pengaturan jaringan yang perlu dipahami sebelum digunakan untuk \textit{virtual screening}.

Terlepas dari pengadaan komputer secara fisik, teknologi \textit{cloud computing} hadir sebagai solusi. Pengguna \textit{cloud computing} tidak perlu memikirkan bagaimana biaya dan perawatan dari teknologi tersebut. Pada saat ini, pemanfaatan aplikasi bersifat \textit{cloud}, dimana pengguna hanya disuguhkan dengan tampilan saja, sedangkan proses komputasi berada di "awan". Sifat virtualisasi yang menjadi andalan \textit{cloud computing} membuat teknologi ini sangat murah untuk menambah komputer secara virtual. \textit{Cloud computing} seolah - olah mampu membuat suatu kumpulan komputer yang saling bekerja sama menyelesaikan \textit{task} seperti \textit{grid computing}. Memaksimalkan virtualisasi merupakan kunci dalam teknologi \textit{cloud computing}.

Teknologi virtualisasi yang ada saat ini terbentur dengan abstraksi komputer pada level \textit{hardware}. Pembuatan komputer virtual tersebut membutuhkan \textit{resource} memori penyimpanan dan kemampuan komputasi yang telah ditentukan sebelumnya. Dengan begitu jumlah komputer virtual akan terbatas dengan \textit{resource} yang dimiliki oleh \textit{host}. Disatu sisi, kinerja virtual komputer virtual tidak akan optimal ketika terdapat komputer virtual yang tidak melakukan apa apa, sehingga alokasi \textit{resource} komputer virtual tersebut dapat dibagi secara merata kepada komputer virtual lainnya. Docker, sebagai \textit{platform} virtualisasi memberikan  pandangan baru dalam mengoptimalkan proses virtualisasi tersebut.  

%-----------------------------------------------------------------------------%
\section{Permasalahan}
%-----------------------------------------------------------------------------%
%-----------------------------------------------------------------------------%
\subsection{Definisi Permasalahan}
%-----------------------------------------------------------------------------%
\hspace{0.5cm}Berdasarkan latar belakang yang telah dijelaskan, penulis berusaha menganalisis apakah pemanfaatan \textit{platform} Docker pada teknologi \textit{cloud computing} dapat diajukan sebagai solusi alternatif dari \textit{supercomputer} dan mengukur \textit{scalability platform} Docker. Untuk itu, penulis akan mencoba \textit{virtual screening} dengan menggunakan aplikasi Autodock dan Autodock Vina yang telah terinstall pada \textit{platform} Docker.

%-----------------------------------------------------------------------------%
\subsection{Batasan Permasalahan}
%-----------------------------------------------------------------------------%
\hspace{0.5cm}Pada penelitian ini, penulis lebih berfokus kepada performa Docker dalam menjalankan aplikasi Autodock dan Autodock Vina untuk \textit{virtual screening}. Penulis tidak akan membahas segi \textit{virtual screening} dikarenakan keterbatasan pengetahuan yang dimiliki dalam \textit{drug discovery}

%-----------------------------------------------------------------------------%
\section{Tujuan}
%-----------------------------------------------------------------------------%
\hspace{0.5cm}Penelitian ini secara umum bertujuan untuk mengenalkan pemanfaatan \textit{cloud computing} dengan \textit{platform} Docker dalam virtualisasi komputer. Diharapkan performa yang diperoleh dapat menyaingi \textit{supercomputer} maupun \textit{grid computing} dalam \textit{virtual screening}.


%-----------------------------------------------------------------------------%
\section{Posisi Penelitian}
%-----------------------------------------------------------------------------%
\hspace{0.5cm}Penelitian ini mencari alternatif lain dari \textit{supercomputer} dengan memanfaatkan teknologi \textit{cloud computing}. Hasil dari penelitian ini akan dibandingkan dengan hasil penelitian serupa yang telah dikerjakan oleh Bapak Muhammad Hafizhuddin Hilman, S.Kom., M.Kom. \cite{cluster_pak hilman} \cite{cloud_pak hilman} 

%-----------------------------------------------------------------------------%
\section{Sistematika Penulisan}
%-----------------------------------------------------------------------------%
Penulisan ini terbagi dalam 5 bab :
\begin{itemize}
	\item BAB 1 \babSatu \\
	Bagian ini berisikan latar belakang, permasalahan, tujuan dan ruang lingkup penelitian, serta sistematika penulisan tugas akhir penelitian. 
	\item BAB 2 \babDua \\
	Bagian ini berisikan penjelasan \textit{drug discovery} dengan cara \textit{virtual screening} memanfaatkan teknologi informasi secara umum. Selain itu, \textit{cloud computing} dan Docker akan dijelaskan secara detail.
	\item BAB 3 \babTiga \\
	Bagian ini berisikan detail tahap-tahap penelitian dan data yang digunakan.
	\item BAB 4 \babEmpat \\
	Bagian ini berisikan hasil dari penelitian yang telah dilaksanakan dan analisis dari hasil yang diperoleh.
	\item BAB 5 \kesimpulan \\
	Bagian ini berisikan kesimpulan penulis terkait dengan penelitian dan saran penulis dalam penelitian ke depannya.
\end{itemize}


