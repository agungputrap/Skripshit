%
% Halaman Abstract
%
% @author  Andreas Febrian
% @version 1.00
%

\chapter*{ABSTRACT}

\vspace*{0.2cm}

\noindent \begin{tabular}{l l p{11.0cm}}
	Name&: & \penulis \\
	Program&: & \program \\
	Title&: & \judulInggris \\
\end{tabular} \\ 

\vspace*{0.5cm}

\noindent Drug discovery is a sequence activities undertaken by pharmaceutical companies. It takes long time in finding the right candidate  drug to cure disease. The role of the computer makes it easy for research, especially in drug discovery. The process of molecular docking with  matching receptor with ligand and virtual screening of the data collection of receptor requires high computational resources. However, such convenience comes with negative terms. Facilities and funding constraints led to the need for alternative solutions in the procurement of high performance computing. With the development of cloud computing, the computing resource use is not an obstacle. Optimization of virtualization in cloud computing may provide a promising solution. Docker as virtualization platform provides a new concept and ease of doing virtualization. In this study addressed the use of Docker to test the performance of applications Autodock version 4.2 and Autodock Vina version 1.1 in carrying out virtual screening process. Results of this research is the analysis of the running time of the application in container.

\vspace*{0.2cm}

\noindent \textbf{Keywords:} cloud computing, drug discovery, virtual screening, virtualization, Docker
\newpage