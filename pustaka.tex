%
% Daftar Pustaka 
% 

% 
% Tambahkan pustaka yang digunakan setelah perintah berikut. 
% 
\begin{thebibliography}{4}

\bibitem{cluster_pak hilman}
{Muhammad H. Hilman.Juni 2010.\f{"Evaluasi Kinerja Autodock 4.2 dan Autodock Vina 1.1 dalam Proses Molecular Docking dan Virtual Screening di Lingkungan Cluster Hastinapura"}.Fakultas Ilmu Komputer,Universitas Indonesia.}

\bibitem{cloud_pak hilman}
{Muhammad H. Hilman.Januari 2012.\f{"Analisis Teknik Data Mining dan Kinerja
		Infrastruktur Komputasi Cloud Sebagai Bagian dari
		Sistem Perancangan Obat Terintegrasi"}.Fakultas Ilmu Komputer,Universitas Indonesia.}

\bibitem{Drugdiscovery}
{Pharma \f{Drug Discovery and Development : UNDERSTANDING THE R and D PROCESS}.February 2007.\url{http://www.phrma.org/sites/default/files/pdf/rd_brochure_022307.pdf}.}

\bibitem{Drugdiscovery2}
{Myers S, Baker A. \f{Drug discovery - an operating model for a new era}.Nat Biotechnol 2001; 19: 727–30}

\bibitem{doCADD}
{Matthew Segall.\f{Can we really do computer-aided drug design?}.J Comput Aided Mol Des (2012) 26:121–124}

\bibitem{reviewCADD}
{Si-sheng OU-YANG, Jun-yan LU, Xiang-qian KONG, Zhong-jie LIANG, Cheng LUO, Hualiang JIANG.\f{Review Computational drug discovery}.Acta Pharmacologica Sinica (2012) 33: 1131–1140}

\bibitem{diagram drug discovery}
{Gambar diagram drug discovery. Diambil dari artikel : Myers S, Baker A. \f{Drug discovery - an operating model for a new era}.Nat Biotechnol 2001; 19: 727–30 }

\bibitem{rankingSBDD}
{Rajamani R, Good AC.2007. \f{"Ranking poses in structure-based lead discovery and optimization: current trends in scoring function development"}.Current Opinion in Drug Discovery and Development 10 (3): 308–15.}

\bibitem{gambar MolecularDocking}
{Gambar ilustrasi molecular docking. Diambil dari artikel : Bachwani Mukesh, Kumar Rakesh.\f{Molecular Docking : A Review}.IJRAP (2011) 2 (6) 1746 - 1751.}

\bibitem{Drugdesign}
{\f{Drug design}. \url{http://strbio.biochem.nchu.edu.tw/classes/special_topics_biochem/course_ppts/rational_drug_design-2014.pdf}.diakses pada 3 Juni 2015 pukul 04:49}

\bibitem{MolecularDocking}
{Bachwani Mukesh, Kumar Rakesh.\f{Molecular Docking : A Review}.IJRAP (2011) 2 (6) 1746 - 1751.}

\bibitem{pharmacophore}
{Contoh model \textit{pharmacophore}.\url{http://en.wikipedia.org/wiki/Pharmacophore#/media/File:PharmacophoreModel_example.svg} diakses 3 Juni 2015 pukul 18:09}

\bibitem{De Novo}
{Schneider G, Fechner U.August 2005. \f{"Computer-based de novo design of drug-like molecules".} Nat Rev Drug Discov 4 (8): 649–63.}

\bibitem{Docking and Scoring}
{Kitchen DB, Decornez H, Furr JR, Bajorath J.2004.\f{"Docking and scoring in virtual screening for drug discovery: methods and applications"}.Nature reviews. Drug discovery 3 (11): 935–49.}

\bibitem{best-fit}
{Wei BQ, Weaver LH, Ferrari AM, Matthews BW, Shoichet BK.2004. \f{"Testing a flexible-receptor docking algorithm in a model binding site"}. J. Mol. Biol. 337 (5): 1161–82.}

\bibitem{autodock most cited}
{Morris.2007,May 04.\f{"AutoDock is the most cited docking software"} \url{http://autodock.scripps.edu/news/autodock-is-the-most-cited-docking-software} diakses 3 Juni 2015 pukul 06:59}

\bibitem{website resmi}
{Website resmi The Scripps Research Institute,
	\url{http://www.scripps.edu/about/index.html} diakses 6 Juni 2015 pukul 14.39}

\bibitem{autodock}
{Garrett M. Morris et al.2009.\f{"AutoDock4 and AutoDockTools4: Automated Docking with Selective Receptor Flexibility"}.J Comput Chem. 2009 December ; 30(16): 2785–2791. doi:10.1002/jcc.21256}

\bibitem{autodockvina}
{Oleg Trott and Arthur J. Olson.2010.\f{"AutoDock Vina: improving the speed and accuracy of docking	with a new scoring function, efficient optimization and multithreading"}.J Comput Chem. 2010 January 30; 31(2): 455–461. doi:10.1002/jcc.21334.}

\bibitem{Logo autodock}
{Logo Autodock. \url{http://www.kdm.wcss.wroc.pl/w/images/thumb/Autodock_logo.gif/200px-Autodock_logo.gif} diakses 3 Juni 2015 pukul 07:49.}

\bibitem{virtual screening}
{Rester, U.July 2008. \f{"From virtuality to reality - Virtual screening in lead discovery and lead optimization: A medicinal chemistry perspective"}. Curr Opin Drug Discov Devel 11 (4): 559–68.}

\bibitem{grid}
{Ian Foster.Juli 2002.\f{"Lecture Note : What is the Grid? A Three Point Checklist }.\url{http://dlib.cs.odu.edu/WhatIsTheGrid.pdf} . diakses 3 Juni 2015 19.00}

\bibitem{Cloud Computing: Overview and Risk Analysis}
{Fatima A. Alali,Chia-Lun Yeh.2012.\f{"Cloud Computing: Overview and Risk Analysis"}.JOURNAL OF INFORMATION SYSTEMS Vol. 26,No. 2 Fall 2012 pp.13–33}

\bibitem{Cloud Computing Research and Development Trend}
{Shuang Zhang, Shufen Zhang, Xuebin Chen, Xiuzhen Huo.2010.\f{"Cloud Computing Research and Development Trend"}.2010 Second International Conference on Future Networks.}

\bibitem{Layanan cloud}
{Layanan cloud. \url{http://upload.wikimedia.org/wikipedia/commons/thumb/b/b5/Cloud_computing.svg/2000px-Cloud_computing.svg.png} diakses 3 Juni 2015 pukul 20.18}

\bibitem{vmm}
{Amir Ali Semnanian.May 2013.\f{"A STUDY ON VIRTUALIZATION TECHNOLOGY AND ITS IMPACT ON COMPUTER HARDWARE"}Department of Computer Engineering and Computer Science California State University, Long Beach.}

\bibitem{Virtualization in Education}
{IBM Global Education White Paper.October 2007. \f{"Virtualization in Education"}.\url{http://www-07.ibm.com/solutions/in/education/download/Virtualization_in_Education.pdf}}

\bibitem{vmware}
{Situs resmi VMWare. \url{http://www.vmware.com/ap}. diakses 12 Juni 2015 pukul 22.49}

\bibitem{virtualbox}
{Situs resmi VirtualBox.\url{https://www.virtualbox.org/}. diakses 12 Juni 2015 pukul 22.55}

\bibitem{LXC}
{Tobby Banerjee.November 2014.\f{"LXC vs LXD vs Docker - Making sense of the rapidly evolving container ecosystem"}.\url{https://www.flockport.com/lxc-vs-lxd-vs-docker-making-sense-of-the-rapidly-evolving-container-ecosystem/}. diakses 3 Juni 2015 pukul 20.45}

\bibitem{libcontainer}
{Steven J. Vaughan-Nichols.11 Juni 2014.\f{"Docker libcontainer unifies Linux container powers"}.\url{http://www.zdnet.com/article/docker-libcontainer-unifies-linux-container-powers/}. diakses 12 Juni 2015 pukul 23.00}

\bibitem{VMware and Docker – Better Together}
{Chris Wolf. August 2014.\f{"VMware and Docker - Better Together"}.\url{http://blogs.vmware.com/cto/vmware-docker-better-together/} diakses 3 Juni 2015 pukul 20.35}

\bibitem{vmmvscontainer}
{Virtual machine vs docker. \url{http://www.rightscale.com/blog/sites/default/files/docker-containers-vms.png} }

\bibitem{Logo Docker}
{Logo Docker. \url{http://getcloudify.org/img/docker.png}}

\bibitem{herbaldb}
{Arry Yanuar, Abdul Mun’im , Akma Bertha Aprima Lagho, Rezi Riadhi Syahdi, Marjuqi Rahmat, and Heru Suhartanto.2011. \f{Medicinal Plants Database and Three Dimensional Structure of
		the Chemical Compounds from Medicinal Plants in Indonesia}.IJCSI International Journal of Computer Science Issues, Vol. 8, Issue 5, No 1, September 2011}

\bibitem{tutorial autodock}
{The Scripps Research Institute, “Tutorial Autodock 4.2”,
	\url{http://autodock.scripps.edu/faqs-help/tutorial} diakses 6 Juni 2015 pukul
	10.29}

\bibitem{Marek Goldmann}
{Marek Goldmann.11 Sept 2014.\f{"Resource management in Docker"}. \url{https://goldmann.pl/blog/2014/09/11/resource-management-in-docker/}. diakses pada 12 Juni 2015 pukul 16.08}

\bibitem{toplistsupercomputer}
{Top 500 List. \url{http://www.top500.org/lists/2014/11/} diakses 3 Juni 2015 pukul 18:11}

\bibitem{Disadvantages of Virtualization, What’s Your Opinion?}
{Dinesh. \f{"Disdvantages of Virtualization,What's Your Opinion"}.\url{http://www.sysprobs.com/disadvantages-virtualization-opinion}}

\bibitem{tutorial vina}
{The Scripps Research Institute, “Tutorial Autodock Vina 1.1 ”,
	\url{http://vina.scripps.edu/tutorial.html} diakses 6 Juni 2015 pukul 10.29}

\bibitem{comparison_dockvina}
{Gambar tabel perbandingan Autodock dan Autodock Vina. \url{http://vina.scripps.edu/}.}

\end{thebibliography}



