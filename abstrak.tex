%
% Halaman Abstrak
%
% @author  Andreas Febrian
% @version 1.00
%

\chapter*{Abstrak}

\vspace*{0.2cm}

\noindent \begin{tabular}{l l p{10cm}}
	Nama&: & \penulis \\
	Program Studi&: & \program \\
	Judul&: & \judul \\
\end{tabular} \\ 

\vspace*{0.5cm}

\noindent \textit{Drug discovery} merupakan runtutan aktivitas yang dilakukan oleh perusahaan farmasi. Dibutuhkan waktu yang cukup lama hingga puluhan tahun dalam menemukan kandidat obat yang tepat untuk menyembuhkan penyakit. Peran serta komputer memberikan kemudahan bagi penelitian khususnya dalam \textit{drug discovery}. Proses \textit{molecular docking} antara \textit{receptor} dengan \textit{ligand} dan \textit{virtual screening} dari kumpulan data \textit{receptor} membutuhkan sumber daya kinerja komputasi yang tinggi. Namun, kemudahan tersebut datang dengan segi negatif. Kendala fasilitas dan dana menyebabkan perlunya solusi alternatif dalam pengadaan sumber daya komputasi tinggi. Dengan berkembangnya \textit{cloud computing}, penggunaan \textit{resource} komputasi tidak menjadi kendala. Optimisasi virtualisasi dalam \textit{cloud computing} dapat memberikan solusi yang menjanjikan. Docker sebagai \textit{platform} virtualisasi memberikan konsep baru dan kemudahan dalam melakukan virtualisasi. Dalam penelitian ini dibahas mengenai pemanfaatan Docker untuk menguji kinerja aplikasi Autodock versi 4.2 dan Autodock Vina versi 1.1 dalam menjalankan proses \textit{virtual screening}. Hasil dari penelitian ini adalah analisis \textit{running time} kedua aplikasi tersebut dalam container



\vspace*{0.2cm}

\noindent \textbf{Kata Kunci:} \textit{cloud computing}, \textit{drug discovery}, \textit{virtual screening}, virtualisasi, Docker

\newpage